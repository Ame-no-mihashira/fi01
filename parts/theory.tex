\documentclass[../main]{subfiles}

\begin{document}
\textbf{Velocidad instantánea}\\
Considérese la velocidad media:
\begin{equation*}
  \overline{v} = \frac{x_f - x_i}{t_f - t_i}
\end{equation*}
Para diferencias de tiempo cada vez más pequeñas:
\begin{align*}
  v_i &= \lim_{\Delta t \to 0} \overline{v}\\
      &= \frac{d}{dt} x
\end{align*}
De tal modo, es obtenida la velocidad instantánea.
Experimentalmente, es obtenida observando al cuerpo en movimiento en dos posiciones muy cercanas. \parencite{book:alonso1970}

\textbf{Aceleración}\\
Aplicando un proceso similar al anterior, para la aceleración media: \parencite{book:alonso1970}
\begin{align*}
  \overline{a} &= \frac{v_f - v_i}{t_f - t_i}\\
  a &= \lim_{\Delta t \to 0} \overline{a}\\
    &= \frac{d}{dt} v
\end{align*}
\end{document}
